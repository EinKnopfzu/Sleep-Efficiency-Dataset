\documentclass[usegeometry=true]{scrartcl}
\usepackage[ngerman]{babel}
\usepackage[T1]{fontenc}
\usepackage{lmodern}
\usepackage[utf8]{inputenc}
\usepackage{hyperref}
\usepackage{amssymb}
% Dimensionen bitte nicht ändern. 
\usepackage[left=2cm, right=2cm, top=2cm, bottom=2cm, bindingoffset=1cm, includeheadfoot]{geometry}
%Zeilenabstand bitte nicht ändern
\usepackage[onehalfspacing]{setspace}

\usepackage[backend=biber,style=numeric,]{biblatex}\addbibresource{literatur.bib}

\begin{document}
% ----------------------------------------------------------------------------
\subject{Projektbericht zum Modul Information Retrieval und Visualisierung Sommersemester 2023}
\title{Visualisierungen zur analyse der Schlaf Effizienz}
%\subtitle{Untertitel}% optional
\author{Mick Stewart Wörner}% obligatorisch
%\date{10.9.2023}
\maketitle% verwendet die zuvor gemachte Angaben zur Gestaltung eines Titels
% ----------------------------------------------------------------------------
% Inhaltsverzeichnis:
%\tableofcontents
% ----------------------------------------------------------------------------
% Gliederung und Text:

\section{Einleitung}
Tipps zu Latex und Koma-Script für Hausarbeiten sind im \href{http://mirrors.ctan.org/info/latex-refsheet/LaTeX_RefSheet.pdf}{LaTeX Reference Sheet for a thesis with KOMA-Script} von Marion Lammarsch und Elke Schubert zusammengefasst. 
Der Bericht fällt in die Kategorie von InfoVis-Paper, die Tamara Munzner Design Study nennt \cite{Munzner2008}: In der Einleitung sollen sie zuerst das Zielproblem beschrieben. Daraus sollen sie Fragestellungen motivieren, die mittels Techniken der Informationsvisualisierung beantwortet werden können. In dem Abschnitt direkt unter der Überschrift Einleitung sollen Sie nach einer kurzen Einleitung Fragestellungen und das Zielproblem motivieren und besschreiben. 

\subsection{Anwendungshintergrund}
Sie müssen genug Hintergrund bereitstellen, so dass die Lesenden sich ein Urteil bilden können, ob ihre Lösung funktioniert. Sie sollen die Lesenden jedoch nicht mit Anwendungsdetails so überschütten, dass der Fokus auf die Fragen zur Informationsvisualisierung untergehen. 
Eine Visualisierung muss
I expressiv:Expresivität bedeutet, dass die Daten unverfälscht wiedergegeben werden. Grundsätzlich sollen nur die Informationen dargestellt werden, die auch im Datenamterial vorhanden sind.
I effektiv (Effektivität hängt nicht nur von den Daten ab, sondern auch
von: 1. dem Bearbeitungsziel und 2. den Fähigkeiten des Betrachters 3. Eine effektive Visualisierung versucht die Inhalte auf intuitiven
Wegen zu präsentierten)
I angemessen sein (Angemessenheit beschreibt den Verbrauch an Ressourcen zur
Erzeugung der Visualisierung)
\subsection{Zielgruppen}
\textbf {Die Visualsierung soll dabei helfen die Einflüsse in der lebensweise zu analysieren. Hierbei werden Lebensstile anhand von Koffeein, alkohol, Tabak Konsum und Sport verhalten analysiert. Dies sol Forschern als auch der privatperson dabei Helfen, risikoreiche verhaltensweisen zu identifizieren um die Lebensqualität zu verbessern: Die Schlafqualität hat einen hohen einfluss auf die Lebensqualität }
Beschreiben sie die Personengruppe oder Personengruppen, die das von ihnen benannte Anwendungsproblem lösen möchte. Auf welches Vorwissen können sie in dieser Gruppen von Anwenderinnen aufbauen? Welche Informations"-bedürf"-nisse werden durch die Visualisierungen adressiert?: The data was then analyzed to understand the relationship between lifestyle factors and sleep patterns and to identify any potential areas for intervention to improve sleep
\subsection{Überblick und Beiträge}
In diesem Abschnitt geben sie einen kurzen Überblick über die Daten und verwendeten Visualisierungen. Dann benennen sie die Beiträge ihres Projekts. Diese Beiträge müssen sie in den hinteren Teilen des Berichts genauer ausführen und belegen.

\section{Daten}
Beschreiben Sie vorhandenen Daten. Gehen sie kritisch darauf ein, in wie weit sich die Daten für die Bearbeitung der Fragestellungen und dem Erreichen von Lösungen für die oben beschriebene Zielgruppen eignen. Haben sie die Daten sinnvoll mit weiteren Datenquellen ergänzt? Wenn ja, wie?
Erklären sie die technische Bereitstellung der Daten.
\textbf { Die Daten }
Wie sind die Daten zugänglich? Welche Formate werden genutzt. Gibt es Besonderheiten beim Lesen der Formate?
Beschreiben sie die Datenvorverarbeitung.
 Welche Datenvorverarbeitungsschritte sind notwendig? Beschreiben Sie die einzelnen Schritte und begründen sie sie, z.B. warum werden manche Daten weggelassen, über welche Mengen werden Durchschnitte berechnet, warum sind die so berechneten Werte aussagekräftiger als andere Werte. Wenn möglich sollen sie die Datenvorverarbeitung in Elm programmieren, so dass ihre Anwendung auf eine Änderung der Rohdaten reagieren kan.
 Realisiert eine Daten-zu-Daten-Abbildung\\
 \textbf { Mögliche Operationen:\\
    - Vervollständigung, Interpolation\\
    - Projektion (Reduzierung der Variablen)\\
    - Selektion (Anwendung von Filterkriterien, Glättung,Ausreißereliminierung)\\
    - Berechnung impliziter Eigenschaften (z.B. Maximum,Gradient)\\
    - Konvertierung\\
    \\
    Die Daten sollten so konvertiert werden, dass Sie den Visualisierungsanwendungen entsprechen. 
 }

\section{Visualisierungen}
\subsection{Analyse der Anwendungsaufgaben}
Analysieren sie die konkreten Anwendungsaufgaben, die die Lösung des Zielproblems durch die Anwender:innen bearbeitet werden müssen. 
Welche sinnvollen mentale Modelle helfen den Personen bei der Bearbeitung. 
%Welche Visualisierungen helfen den Personen, die die Software verwenden, sinnvolle mentale Modelle aufzubauen. 
Sind diese mentalen Modelle für sie notwendig, um die Aufgaben lösen zu können? Gehen sie bei ihrer Argumentation von den Anwendungsaufgaben aus und kommen sie dann zu den mentalen Modellen, deren Aufbau durch Visualisierungen unterstützt wird. 
\subsection{Anforderungen an die Visualisierungen}
Leiten sie Anforderungen an das Design der Visualisierungen ab, die sich durch ihre Analyse des Zielproblems ergeben.
\subsection{Präsentation der Visualisierungen}
Präsentieren sie die visuelle Abbildungen und Kodierungen der Daten und Interaktionsmöglichkeiten. 
Sie müssen  begründen, warum und wie gut ihre Designentscheidungen die erstellten Anforderungen erfüllen. 
Weiterhin müssen sie begründen, warum die gewählte visuelle Kodierung der Daten für das zulösenden Problem passend ist.
Typische Argumente würden hier auf Wahrnehmungsprinzipien und Theorie über Informationsvisualisierung verweisen. 
Die besten Begründungen diskutieren explizit die konkrete Auswahl der Visualisierungen im Kontext von mehreren verschiedenen Alternativen. 
Machen sie hier nicht den Fehler, einfach nur Visualisierung aus den vorgegebenen Bereichen zu diskutieren, weil das in der Regel nicht sinnvoll ist.
Wenn sie sich für einen Scatterplot entschieden haben, ist ein Zeitreihendiagramm in der Regel keine Alternative.
Diskutieren sie also nicht einfach Zeitreihendiagramme, weil sie in den Anforderungenen an das Projekt neben Scatterplots stehen, sondern suchen sie nach echten alternativen Visualisierungen, die zum Aufbau eines vergleichbaren mentalen Modells führen. 
Diskutieren sie die Expressivität und die Effektivität der einzelnen Visualisierungen. 

Die eben beschriebenen Präsentationen und Begründungen sollen für jede der drei folgenden Visualisierungen durchgeführt werden. 
\subsubsection{Visualisierung Eins}
\subsubsection{Visualisierung Zwei}
\subsubsection{Visualisierung Drei}

\subsection{Interaktion}
Die präsentierten Visualisierungstechniken müssen interaktiv zu einer Anwendung verknüpft werden.
Die Interaktion mit einer Visualisierung soll in den anderen Visualisierungen zu einer Änderung führen. 
Erklären sie die möglichen Interaktionen mit den einzelnen Visualisierungen und die möglichen Verknüpfungen zwischen ihnen. Begründen Sie warum die konkreten Interaktionen umgesetzt wurden und welche Zwecke für die Anwenderinnen mit ihnen unterstützt werden. Begründen sie ebenfalls warum sie andere Interaktionsmöglichkeiten nicht umgesetzt haben. Wenn sie keine der geforderten Interaktionen umsetzen, erhalten Sie im gesamten Projekt deutlichen Punktabzug. 

\section{Implementierung}
Beschreiben Sie die Implementierung ihrer Visualisierungsanwendung in Elm. Stellen die Gliederung ihres Quellcodes vor. Haben Sie verschiedene Elm-Module erstellt. Was war aufwändig umzusetzen, was ließ sich mit dem vorhanden Code aus den Übungen relativ einfach umsetzen? 

Wie sieht die Elm-Datenstruktur für das Model aus, in dem die verschiedenen Zustände der Interaktion gespeichert werden können.

\section{Anwendungsfälle}
Präsentieren sie für jede der drei Visualisierungen einen sinnvollen Anwendungsfall in dem ein bestimmter Fakt, ein Muster oder die Abwesenheit eines Musters visuell festgestellt wird. Begründen sie warum dieser Anwendungsfall wichtig für die Zielgruppe der Anwenderinnen ist. Diskutieren sie weiterhin, ob die oben beschriebene Information auch mit anderen Visualisierungstechniken hätte gefunden werden können. Falls dies möglich wäre, vergleichen sie die den Aufwand und die Schwierigkeiten ihres Ansatzes und der Alternativen. 
\subsection{Anwendung Visualisierung Eins}
\subsection{Anwendung Visualisierung Zwei}
\subsection{Anwendung Visualisierung Drei}

\section{Verwandte Arbeiten}
Führen sie eine kurze Literatursuche in der wissenschaftlichen Literatur zu Informationsvisualisierung und Visual Analytics nach ähnlichen Anwendungen durch. Diskutieren sie mindestens zwei Artikel. Stellen sie Gemeinsamkeiten und Unterschiede dar.

\section{Zusammenfassung und Ausblick}
Fassen sie die Beiträge ihre Visualisierungsanwendung zusammen. Wo bietet sie für die Personen der Zielgruppe einen echten Mehrwert.

Was wären mögliche sinnvolle Erweiterungen, entweder auf der Ebene der Visualisierungen und/oder auf der Datenebene?

\section*{Anhang: Git-Historie}

\printbibliography

\end{document}

