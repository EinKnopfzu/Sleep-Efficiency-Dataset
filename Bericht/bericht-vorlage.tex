\documentclass[usegeometry=true]{scrartcl}
\usepackage[ngerman]{babel}
\usepackage[T1]{fontenc}
\usepackage{lmodern}
\usepackage[utf8]{inputenc}
\usepackage{hyperref}
\usepackage{amssymb}
\usepackage{graphicx}
% Dimensionen bitte nicht ändern. 
\usepackage[left=2cm, right=2cm, top=2cm, bottom=2cm, bindingoffset=1cm, includeheadfoot]{geometry}
%Zeilenabstand bitte nicht ändern
\usepackage[onehalfspacing]{setspace}

\usepackage[backend=biber,style=numeric,]{biblatex}\addbibresource{literatur.bib}

\begin{document}
% ----------------------------------------------------------------------------
\subject{Projektbericht zum Modul Information Retrieval und Visualisierung Sommersemester 2023}
\title{Visualisierungen zur Analyse von Einflussfaktoren auf die Schlaff Effizienz}
%\subtitle{Untertitel}% optional
\author{Mick Stewart Wörner}% obligatorisch
%\date{10.9.2023}
\maketitle% verwendet die zuvor gemachte Angaben zur Gestaltung eines Titels
% ----------------------------------------------------------------------------
% Inhaltsverzeichnis:
%\tableofcontents
% ----------------------------------------------------------------------------
% Gliederung und Text:

\section{Einleitung}
Tipps zu Latex und Koma-Script für Hausarbeiten sind im \href{http://mirrors.ctan.org/info/latex-refsheet/LaTeX_RefSheet.pdf}{LaTeX Reference Sheet for a thesis with KOMA-Script} von Marion Lammarsch und Elke Schubert zusammengefasst. 
Der Bericht fällt in die Kategorie von InfoVis-Paper, die Tamara Munzner Design Study nennt ~\cite{Munzner2008}: In der Einleitung sollen sie zuerst das Zielproblem beschrieben. Daraus sollen sie Fragestellungen motivieren, die mittels Techniken der Informationsvisualisierung beantwortet werden können. In dem Abschnitt direkt unter der Überschrift Einleitung sollen Sie nach einer kurzen Einleitung Fragestellungen und das Zielproblem motivieren und besschreiben. 

\subsection{Anwendungshintergrund}
Sie müssen genug Hintergrund bereitstellen, so dass die Lesenden sich ein Urteil bilden können, ob ihre Lösung funktioniert. 
Sie sollen die Lesenden jedoch nicht mit Anwendungsdetails so überschütten, dass der Fokus auf die Fragen zur Informationsvisualisierung untergehen. 
Eine Visualisierung muss 
\textbf{expressiv: Expresivität bedeutet, dass die Daten unverfälscht wiedergegeben werden. Grundsätzlich sollen nur die Informationen dargestellt werden, die auch im Datenamterial vorhanden sind.\\
I effektiv (Effektivität hängt nicht nur von den Daten ab, sondern auch von:\\ 1. dem Bearbeitungsziel und \\2. den Fähigkeiten des Betrachters \\
3. Eine effektive Visualisierung versucht die Inhalte auf intuitivenWegen zu präsentierten)
I angemessen sein (Angemessenheit beschreibt den Verbrauch an Ressourcen zur Erzeugung der Visualisierung)}
\subsection{Zielgruppen}
\textbf{Die Schlafqualität hat einen signifikanten positiven Einfluss auf die Lebensqualität von Menschen.  \cite*[The association between sleep quality and quality of life: a population-based study Sujin Lee a , Ji Hyun Kim c , Jae Ho Chung b, ]{2}  
Das Menschliche schlafsystem ist allerdings ein hochkompliziertes System das von Multiplen externen Faktoren beeinflusst werden kann.
Die in diesem Projekt dargestellten Visualisierungen sollen dabei helfen Einflüsse und Zusammenhänge bei der Schlafqualität zu erkennen und Forschern sowie Privatpersonen dabei zu helfen den Einfluss von Lebensstilen auf verschiedene Maße von Schlafqualität zu erkennen.
Die untersuchten Attribute sind der Koffeein, Alkohol, Tabak Konsum und Sport sowie das Alter und Geschlecht der Personen.
Dies sollte Personen hefen die Verhaltensweisen zu identifizieren, mit denen Sie den größten Einfluss auf Ihre Schlafqualität haben könnten. 
Dabei kann ausgegengen werden, dass die Benutzer wissen, dass die Datenlage kritisch in hinblick auf die Aussagekräftigkeit und praktische implikationen reflektiert werden muss. 
Dies ist wichtig, da der benutze Datensatz  n < XXXX Datenpunkte besetitz, welche durch den eingebauten Filter weiter reduziert werden können. ab n<30 ist die Statistische Aussagekraft ncihtmehr gegeben.  klein ist.  }
Beschreiben sie die Personengruppe oder Personengruppen, die das von ihnen benannte Anwendungsproblem lösen möchte.
 Auf welches Vorwissen können sie in dieser Gruppen von Anwenderinnen aufbauen? 
 Welche Informations"-bedürf"-nisse werden durch die Visualisierungen adressiert?:
\textbf{Überarbeiten: \\ The data was then analyzed to understand the relationship between lifestyle factors 
  and sleep patterns and to identify any potential areas for intervention to improve sleep \\}
\subsection{Überblick und Beiträge}
\textbf{Im diesem Abschnitt wird eine Überblick auf die Daten und verwendeten Visualisierungstechniken. }In diesem Abschnitt geben sie einen kurzen Überblick über die Daten und verwendeten Visualisierungen. Dann benennen sie die Beiträge ihres Projekts. Diese Beiträge müssen sie in den hinteren Teilen des Berichts genauer ausführen und belegen.

\section{Daten}
Beschreiben Sie vorhandenen Daten. 
\textbf{Der verwendete Datensatz besteht aus: 452 Personen, welche durch Ihre ID identifiziert werden. Ob die ID nur den Datenpunkt oder die Person Identifiziert ist unklar. Daher lässt sich nicht sagen ob Schlafdaten einer Person mehrfach erfasst weurden sind. 
Auf der Kaggle Seite wurde nach angaben des Authors erwähnt, dass der Datensatz im Kontext einer Studie von der ENSIAS, Marroco gesammelt wurde.
Innerhalb einer eingeschränkten Recherche konnten weder auf der Webseite der ENSIAS noch in weitergehender Literaturrecherche eine Quelle identifiziert werden. Daher sollten die Daten und daraus entwickelten Ergebnisse, nicht unreflektiert übernommen werden. 
Der Datensatz hat 18 Attribute. 
 Umgang mit den leeren Feldern-> Mit List.map (Bei einem Nothing, wird der Tupel komplett gelöscht. ), da wir somit nur mit einem sauberem Datensatz arbeiten. Datensatz hat nur Fehlende werte bei gewisse Datenpunkten. (Koffeein, Alkohol, Tabak, Sport).
\\ ID\@ : Identifikator der eine Reihe eindeutig Identifiziert. -> Unique identifier keine Person wird zweimal erfasst. -> keine Zeitentwicklung. -> 452 Personen erfasst. Type : Integer.
\\ Age: Gibt das Alter der Person zum zeitpunkt der erfassung hatte.-> Type : Integer   Datenreichweite 9-69 Jahre. Diskreter Wert.
\\ Gender: Gibt das Geschlecht der Person mittels ``Male'' oder ``Female'' wieder. Type : String : Boolean Wert.
\\ Bedtime: Gibt die Uhrzeit an zu der die Person ins Bett gegangen ist. Type : DateTime: Diskreter Wert. Halbe Stunde Schritte
\\ WakeUp Time = Gibt das Datum und die Uhrzeit an an dem die Person erwacht ist.  Type : DateTime
\\ Sleep Duration = Gibt die Stunden an, die die Person im Bett schlafend verbracht hat in Stunden. Type: Float 
\\  Schlaf Effizienz = Gibt die Zeit an die im Bett schlafend verbracht wurde. Type: Float -> In Prozent -> Interpretation nur in combination mit Anzahl erwacht möglich. -> MICK müssen wir das noch Transmutieren? -> Nacher drüber nachdenken.
\\ REM Sleep percentage = Gibt den Prozentualen Anteil an, der im REM Schlaf verbracht wurde. Type: Integer WICHTIG: In Prozent. 
\\ Deep sleep percentage? = Gibt den Prozentualen Anteil an, der im Tiefschlaf verbracht wurde. Type: Integer WICHTIG : In Prozent.
\\ Light sleep percentage = Gibt den Prozentualen Anteil an, der im Leichtschlaf verbracht wurde. Type: Integer WICHTIG: In Prozent.
\\ Awakenings = Absolute Anzahl an, wie oft eine Person aufgewacht ist. Interpretation -> 0.0 bedeutet, ungestörter Schlaf. -> Lässt an der validität des Datensatzes zweifeln. Da, bei "0.0" mal aufgewacht die person im Schlaf gestorben ist.
\\ Caffeine Intake = Gibt an wie viel Koffeein die Person zu sich genommen hat in den letzten 24 Studenen (In mg) ( Frage für Mick: Halbwertszeit von Koffein? (Adenosin rezeptoren block.) -> Problematik bei der Interpretation. Hoher kofeein intake in the Morining might not have th esame effect as the koffeein intake in the evening, Surge of adenosin before going to bed -> More sleep? Or maybe worse sleep?: Datentyp: VERHALTEN ). 
\\ Alcohol Intake = Gibt an wie viel Alkohol die Personen ind en letzten 24 Stundne zu sich genommen hat. (In OZ)  Maßeinheit. Alkohol-> Nerven gift -> Blockierung der REM Tiefschlafphase? 
\\ Tobacco Intake = Gibt an wie viel Tabak die Person zu sich genommen hat. ->Boolean Wert. Raucht die Person ja oder nein. (Datensatz wurde angeblich in einem Paper produziert.)
\\ Exercise Intake = Gibt an wie viel Sport die Person getrieben hat. Sporteinheiten pro Woche. (Gibt nur die Absolute Anzahl an, das Problem liegt darin, dass die Daten nicht nach Minute aufgeteilt sind. ) \\ }
Gehen sie kritisch darauf ein, in wie weit sich die Daten für die Bearbeitung der Fragestellungen und dem Erreichen von Lösungen für die oben beschriebene Zielgruppen eignen.
\textbf{es ist nciht klar ob die Daten legitim sind. Die Quelle auf Kaggle ist nicht angegeben.-> 
Eignet sich besonders gut für unseren Zweck, da wir die Datenverteilung und das Verhalten von der Variablen zueinander untersuchen wollen.
Andere Einflussfaktoen werden nicht aufgenommen, Datensatz ist etwas klein. Messeinheiten sind nciht ordentlich erfssst.  }
 Haben sie die Daten sinnvoll mit weiteren Datenquellen ergänzt? Wenn ja, wie?
Erklären sie die technische Bereitstellung der Daten.

\textbf{Daten werden als amerikanisches CSV auf Kaggle.com bereitgestellt. (MICK: Genauen Typ der CSV herausfinden. (trennung mit , statt dem Europäischem ;))}

\textbf{ Die Daten }
Wie sind die Daten zugänglich? Welche Formate werden genutzt. Gibt es Besonderheiten beim Lesen der Formate?
Beschreiben sie die Datenvorverarbeitung.
 Welche Datenvorverarbeitungsschritte sind notwendig?
 \textbf{ Einspielen des CSV als String, -> Werden aus dme Github Repository gezogen.  }
  Beschreiben Sie die einzelnen Schritte und begründen sie sie, z.B. warum werden manche Daten weggelassen, über welche Mengen werden Durchschnitte berechnet, warum sind die so berechneten Werte aussagekräftiger als andere Werte. Wenn möglich sollen sie die Datenvorverarbeitung in Elm programmieren, so dass ihre Anwendung auf eine Änderung der Rohdaten reagieren kan.
 
 
  Realisiert eine Daten-zu-Daten-Abbildung\\
 \textbf{ Mögliche Operationen:\\
    - Vervollständigung, Interpolation\\
    - Projektion (Reduzierung der Variablen)\\
    - Selektion (Anwendung von Filterkriterien, Glättung,Ausreißereliminierung)\\
    - Berechnung impliziter Eigenschaften (z.B. Maximum,Gradient)\\
    - Konvertierung\\
    \\}

\textbf{ Die Daten sollten so konvertiert werden, dass Sie den Visualisierungsanwendungen entsprechen. Veränderung der Merkmalsausprägungs Daten alle Strings außer ID in eine Interpretierbaren Wert. Unsere Strings sind Boolean, male oder female und Raucher Yes und No }
   

\section{Visualisierungen}
\subsection{Analyse der Anwendungsaufgaben}
Die Aufgaben die durch die Visualisierung gelöst werden sollen: \\ 
\textbf{ Das Problem ist folgendes: In hochkomplexen Systemen sind die Einflussfaktoren
 die die Ausprägung eines Merkmals beeinflussen nicht immer klar. Daher ist es wichtig, 
 dass wir die Daten visualisieren um die Zusammenhänge zu erkennen.\\
Die Visualisierung soll uns also erlauben den Einfluss von multiplen Verhaltensindikatoren
 auf die Ausprägung eines Merkmales zu schätzen. 
Die klassische Herangehensweise in den Einfluss zu überprüfen sind hochkomplizierte und benötigen
 statistisches Hintergrundwissen auf Seiten des Anwenders. Die Visualisierungsanwendung soll es dem Anwender ermöglichen
 den Datensatz und die Merkmale derer zu untersuchen und Rückschlüsse auf die Beziehungen von Ausprägungen untereinander
  und mit Verhaltensindikatoren und zu treffen. Dies ermöglicht dem Anweder die Daten auf Validität zu überprüfen und die  \\  }
Analysieren sie die konkreten Anwendungsaufgaben, die die Lösung des Zielproblems durch die Anwender:innen bearbeitet werden müssen. 

Welche sinnvollen mentale Modelle helfen den Personen bei der Bearbeitung. 
\textbf{Aufgabenstellung: Analyse der Variablen und den Einfuss der Verhaltensindikatoren}
%Welche Visualisierungen helfen den Personen, die die Software verwenden, sinnvolle mentale Modelle aufzubauen. 
Sind diese mentalen Modelle für sie notwendig, um die Aufgaben lösen zu können? 
Gehen sie bei ihrer Argumentation von den Anwendungsaufgaben aus und kommen sie dann zu den mentalen Modellen, deren Aufbau durch Visualisierungen unterstützt wird. 
\subsection{Anforderungen an die Visualisierungen}
Leiten sie Anforderungen an das Design der Visualisierungen ab, die sich durch ihre Analyse des Zielproblems ergeben.

\textbf{ Anforderungen an das Design. }
\subsection{Präsentation der Visualisierungen}
Präsentieren sie die visuelle Abbildungen und Kodierungen der Daten und Interaktionsmöglichkeiten. 
Sie müssen  begründen, warum und wie gut ihre Designentscheidungen die erstellten Anforderungen erfüllen. 
Weiterhin müssen sie begründen, warum die gewählte visuelle Kodierung der Daten für das zulösenden Problem passend ist.
Typische Argumente würden hier auf Wahrnehmungsprinzipien und Theorie über Informationsvisualisierung verweisen. 
Die besten Begründungen diskutieren explizit die konkrete Auswahl der Visualisierungen im Kontext von mehreren verschiedenen Alternativen. 
Machen sie hier nicht den Fehler, einfach nur Visualisierung aus den vorgegebenen Bereichen zu diskutieren, weil das in der Regel nicht sinnvoll ist.
Wenn sie sich für einen Scatterplot entschieden haben, ist ein Zeitreihendiagramm in der Regel keine Alternative.
Diskutieren sie also nicht einfach Zeitreihendiagramme, weil sie in den Anforderungenen an das Projekt neben Scatterplots stehen, sondern suchen sie nach echten alternativen Visualisierungen, die zum Aufbau eines vergleichbaren mentalen Modells führen. 
Diskutieren sie die Expressivität und die Effektivität der einzelnen Visualisierungen. 

Die eben beschriebenen Präsentationen und Begründungen sollen für jede der drei folgenden Visualisierungen durchgeführt werden. 
\subsubsection{Visualisierung Eins: QQ-Plot Datenlage Normalverteilt?}

\subsubsection{Visualisierung Zwei: Boxplott}
Ziel der Visualisierung: 
Erklärung, wie diese aufgebaut sind. Daher 
\subsubsection{Visualisierung Drei: Force Graph.}
Ziel der Visualisierung: 
Erklärung, wie diese aufgebaut sind. 
\subsection{Interaktion}
Die präsentierten Visualisierungstechniken müssen interaktiv zu einer Anwendung verknüpft werden.
Die Interaktion mit einer Visualisierung soll in den anderen Visualisierungen zu einer Änderung führen. 
Erklären sie die möglichen Interaktionen mit den einzelnen Visualisierungen und die möglichen Verknüpfungen zwischen ihnen. Begründen Sie warum die konkreten Interaktionen umgesetzt wurden und welche Zwecke für die Anwenderinnen mit ihnen unterstützt werden. Begründen sie ebenfalls warum sie andere Interaktionsmöglichkeiten nicht umgesetzt haben. Wenn sie keine der geforderten Interaktionen umsetzen, erhalten Sie im gesamten Projekt deutlichen Punktabzug. 

\section{Implementierung}
Beschreiben Sie die Implementierung ihrer Visualisierungsanwendung in Elm. Stellen die Gliederung ihres Quellcodes vor. Haben Sie verschiedene Elm-Module erstellt. Was war aufwändig umzusetzen, was ließ sich mit dem vorhanden Code aus den Übungen relativ einfach umsetzen? 

Wie sieht die Elm-Datenstruktur für das Model aus, in dem die verschiedenen Zustände der Interaktion gespeichert werden können.

\section{Anwendungsfälle}
Präsentieren sie für jede der drei Visualisierungen einen sinnvollen Anwendungsfall 
in dem ein bestimmter Fakt, ein Muster oder die Abwesenheit eines Musters visuell festgestellt wird.
Begründen sie warum dieser Anwendungsfall wichtig für die Zielgruppe der Anwenderinnen ist.
Diskutieren sie weiterhin, ob die oben beschriebene Information auch mit anderen 
Visualisierungstechniken hätte gefunden werden können.
Falls dies möglich wäre, vergleichen sie die den Aufwand und die Schwierigkeiten ihres Ansatzes und der Alternativen. 
\subsection{Anwendung Visualisierung Eins}

\subsection{Anwendung Visualisierung Zwei}
\subsection{Anwendung Visualisierung Drei}

\section{Verwandte Arbeiten}
Führen sie eine kurze Literatursuche in der wissenschaftlichen Literatur zu Informationsvisualisierung und Visual Analytics nach ähnlichen Anwendungen durch. Diskutieren sie mindestens zwei Artikel. Stellen sie Gemeinsamkeiten und Unterschiede dar.

\section{Zusammenfassung und Ausblick}
Fassen sie die Beiträge ihre Visualisierungsanwendung zusammen. Wo bietet sie für die Personen der Zielgruppe einen echten Mehrwert.

Was wären mögliche sinnvolle Erweiterungen, entweder auf der Ebene der Visualisierungen und/oder auf der Datenebene?

\section*{Anhang: Git-Historie}

\printbibliography

\end{document}

